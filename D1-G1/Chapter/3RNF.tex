\section{Requisiti Non funzionali}

\textbf{\labelText{RNF1}{label:RNF1}. Compatibilità}
\begin{itemize}
    \item L’app deve funzionare su iOS (9 o superiori) e Android 7 (o superiori); 
\end{itemize}
\textbf{\labelText{RNF2}{label:RNF2}. Localizzazione}
\begin{itemize}
    \item \textcolor{red}{Il sistema utilizza il GPS del dispositivo};
\end{itemize}
\textbf{\labelText{RNF3}{label:RNF3}. Memorizzazione dati}
\begin{itemize}
    \item I dati dell’utente devono essere messi in un \textbf{database} e devono essere \textbf{crittografati} utilizzando i software forniti da Amazon Relational Database Service;
\end{itemize}
\textbf{\labelText{RNF4}{label:RNF4}. Privacy}
\begin{itemize}
    \item L’applicazione deve seguire le normative imposte dal \textbf{GDPR}, per i motivi spiegati in “failure rate”;
\end{itemize}
\textbf{\labelText{RNF5}{label:RNF5}. Lingua di sistema}
\begin{itemize}
    \item La lingua deve essere la stessa del \textbf{sistema operativo};
\end{itemize}
\textbf{\labelText{RNF6}{label:RNF6}. Autorizzazioni}
\begin{itemize}
    \item \textcolor{red}{L’applicazione all’avvio dovrà richiedere l’accesso al GPS del dispositivo, nel caso non venga fornita l’applicazione si chiude};
\end{itemize}
\textbf{\labelText{RNF7}{label:RNF7}. Performance}
\begin{itemize}
    \item la ricerca deve dare una risposta in un massimo di 10 secondi (nel peggiore dei casi e condizioni di connessione). Inoltre l’applicazione non deve impiegare più di 5 secondi all’avvio e il movimento tra menù dev’essere fluido con animazione di scorrimento e veloce(inferiore a 1 secondo);
    \item \underline{failure rate}: gli errori durante il pagamento della carta di credito non devono superare l’1\%. Questa scelta è dovuta all’immagine dell’applicazione in quanto gestisce dati sensibili;
    \item \underline{capacità}: l’app non deve pesare più di 250 MB;
\end{itemize}
\textbf{\labelText{RNF8}{label:RNF8}. Ricerca}
\begin{itemize}
    \item RNF 8.1 legato a \ref{itm:RF6}: i colori dei parcheggi in base alla disponibilità sono: \textbf{rosso} (meno del 10\%), \textbf{arancione} (tra 10 e 30\%) e \textbf{verde} (più del 30\%);

    \item RNF 8.2 legato a \ref{itm:RF6}: default della ricerca: prezzo crescente e \textbf{5 km} di raggio dalla posizione;
\end{itemize}
\textbf{\labelText{RNF9}{label:RNF9}. Registrazione ed inserimento dati carta}
\begin{itemize}
    \item RNF 9.1 legato a \ref{itm:RF1}: i dati personali hanno un tetto massimo di caratteri per motivi di \textbf{sicurezza}: nome massimo 30 caratteri, cognome massimo 30, data di nascita in formato gg/mm/aaaa. Inoltre nome e cognome possono essere composti da soli caratteri alfabetici. Il codice fiscale è costituito da un’espressione alfanumerica di 16 caratteri. Numero di telefono: deve avere 10 cifre;
    \item RNF 9.2 legato a \ref{itm:RF1}, \ref{itm:RF2} e \ref{itm:RF12}: caratteristiche della password: lunghezza almeno 9 caratteri, una maiuscola, un numero, un carattere speciale e non deve avere ripetizioni (esempio 1-1-1) e sequenze (esempio 1-2-3-4);
    \item RNF 9.3 legato a \ref{itm:RF4} e \ref{itm:RF10}: L’utente deve inserire: nome, cognome, numero carta, data di scadenza e codice cvv. La carta di credito deve essere verificata: numero di carta deve essere una sequenza 16 cifre. la prima cifra della sequenza deve essere 4 (in caso di carta visa) o 5(in caso mastercard). La data di scadenza di formato mm/aa. e numero cvv di 3 cifre;
\end{itemize}
\textbf{\labelText{RNF10}{label:RNF10}. Login}
\begin{itemize}
    \item RNF 10.1 legato \ref{itm:RF2}: se l’utente sbaglia password per 3 volte di fila l’\textbf{account viene bloccato} e viene inviata una mail che riferisce l’accaduto con un link per reimpostare la password;
    \item RNF 10.2 legato \ref{itm:RF3} il sistema controlla nel database se la password inserita è associata alla mail inserita;
\end{itemize}
\textbf{\labelText{RNF11}{label:RNF11}. Metodi di pagamento}
\begin{itemize}
    \item RNF 11.1 legato a \ref{itm:RF23}: il \textbf{metodo di pagamento principale} è il wallet, nel caso non si possa pagare per motivi già specificati, si procede con la carta di credito:
    \item RNF 11.2 legato a \ref{itm:RF10}: sono accettate solo carte con circuito visa e mastercard.
\end{itemize}
\textbf{\textcolor{red}{\labelText{RNF12}{label:RNF12}. Prenotazione}}
\begin{itemize}
    \item \textcolor{red}{Nel caso in cui la prenotazione fosse per un periodo minore di 24h, il sistema scala \textbf{immediatamente} il numero di parcheggi totali disponibili.}
    \item \textcolor{red}{Nel caso in cui la prenotazione fosse per un periodo maggiore di 24h, il sistema scala il numero di parcheggi totali disponibili \textbf{48h prima dell'orario d'arrivo atteso}. Se invece la prenotazione viene effettuata prima delle 48h dall'arrivo previsto, il contatore viene decrementato \textbf{istantaneamente}.}
\end{itemize}