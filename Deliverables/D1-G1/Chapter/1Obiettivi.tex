\section{Obiettivi di Progetto}

    Il progetto ha come obiettivo la realizzazione di un sistema per la gestione ottimizzata dei rifiuti nella citt\`a di Trento, un'applicazione mobile per i cittadini e una web app per il comune. \\Lo scopo è fornire strumenti utili sia ai cittadini che agli operatori comunali per migliorare la raccolta e lo smaltimento dei rifiuti.
    \\ \\In dettaglio, il sistema permette di:

\subsection{Monitoraggio e Ottimizzazione della Raccolta Rifiuti}

    Il \textbf{cittadino} può accedere ad una \textbf{mappa interattiva} che gli permetta di vedere in tempo reale il livello di riempimento dei cassonetti nella sua zona. Inoltre, può consultare il \textbf{calendario} della pulizia stradale, visualizzare la \textbf{posizione} degli ecopunti temporanei per la raccolta di rifiuti speciali e le aree predisposte alla consegna di vuoti in vetro o plastica.
    \\L' \textbf{operatore ecologico} può accedere ad una \textbf{mappa} \textbf{interattiva} che evidenzi i cassonetti con livelli di riempimento
    elevati, permettendogli di seguire percorsi per massimizzare l’efficienza della raccolta e ridurre i tempi di lavoro.

\subsection{Informazioni sulla Raccolta Rifiuti}

    Il \textbf{cittadino} può ricevere un \textbf{promemoria} per la raccolta porta a porta, in modo da sapere con precisione quando e dove lasciare i rifiuti. Quando un cassonetto vicino alla sua abitazione è pieno, il cittadino riceve una \textbf{notifica} che lo informa della situazione dello stato.\\Al fine di \textbf{facilitare la} \textbf{pulizia stradale}\textbf{ }ed evitare multe, il cittadino può consultare la mappa per informarsi sulle zone interessate e spostare l'auto.
    \\Il \textbf{cittadino} può prenotare lo smaltimento dei rifiuti presso un \textbf{ecocentro} e simulare il \textbf{calcolo della TARI}, scegliendo tra diverse modalità.

\subsection{Segnalazione delle Aree Inquinate}

    Il \textbf{cittadino} deve poter segnalare le zone in cui sono presenti \textbf{rifiuti abbandonati}. Deve avere la possibilità di allegare foto alla \textbf{segnalazione} per permettere agli operatori ecologici di intervenire in modo mirato.
    \\Queste segnalazioni sono accessibili nella \textbf{dashboard}, accessibile dal \textbf{comune} per prendere provvedimenti.


\subsection{Analisi Dati Intelligente}

    Il \textbf{comune} deve poter accedere a una \textbf{dashboard} per monitorare quali zone registrano un \textbf{riempimento} più rapido dei \textbf{cassonetti}, permettendo di valutare l'aggiunta di nuove infrastrutture ecologiche.\\
    Attraverso l’\textbf{analisi} di dati storici, il sistema prevede l’andamento della \textbf{produzione di rifiuti}.\\Queste informazioni consentiranno l'ottimizzazione e la pianificazione degli interventi di raccolta.
    \\Infine, il \textbf{comune} può monitorare anche la gestione dei \textbf{cestini stradali}, individuando quali richiedono svuotamenti più frequenti.