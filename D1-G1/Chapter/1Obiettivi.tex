\section{Obiettivi}

Il progetto ha come obiettivo la realizzazione di un sistema digitale integrato per la gestione ottimizzata dei rifiuti nella citt\`a di Trento. Il sistema sar\`a costituito da un'applicazione mobile (per i cittadini) e una web app (per il comune), mirate a fornire strumenti utili sia ai cittadini che agli operatori comunali per migliorare la raccolta e lo smaltimento dei rifiuti.

In dettaglio, il sistema dovr\`a permettere:

\subsection{Cittadini}
\begin{itemize}
    \item \textbf{Monitoraggio in tempo reale}: mappa interattiva che mostra: 
    \begin{itemize}
        \item il livello di riempimento dei cassonetti nelle vicinanze
    \end{itemize}
    \begin{itemize}
        \item schedule della pulizia stradale
        \item ecopunti temporanei di raccolta rifiuti speciali
        \item aree in cui consegnare i vuoti in vetro/plastica
    \end{itemize}
    \item \textbf{Prenotazione ritiro rifiuti speciali}: possibilit\`a di prenotare, tramite app, il prelievo di rifiuti speciali come ingombranti o pericolosi.
    \item \textbf{Informazioni sui punti di raccolta temporanei}: elenco e localizzazione di punti per la raccolta di materiali specifici (es. olio esausto, batterie, ecc.).
    \item \textbf{Notifiche intelligenti}:
    \begin{itemize}
        \item Avviso quando un cassonetto vicino \`e pieno.
        \item Promemoria per la raccolta porta a porta.
        \item Avviso di turno di pulizia stradale, con lo scopo di spostare autovetture di proprietà sostanti a lato della carreggiata, onde evitare spiacevoli e salatissime multe.
    \end{itemize}
    \item \textbf{Segnalazioni da parte degli utenti}:
    \begin{itemize}
        \item Identificazione di zone con rifiuti abbandonati e zone particolarmente sporche dove servono più cestini/cassonetti, con la possibilità di allegare foto/video segnaletici.
    \end{itemize}
    \item \textbf{Sezione informativa}: tutorial e guide per favorire la corretta gestione dei rifiuti.
    \item \textbf{Sezione informazioni }\textbf{ecocentri}: orari di apertura, indirizzi, prenotazione, tipo di rifiuti che possono essere consegnati e controllo saturazione per ognuno di essi.
\end{itemize}

\subsection{Comune e operatori ecologici}
\begin{itemize}
    \item \textbf{Ottimizzazione dei percorsi di raccolta}: i netturbini potranno visualizzare il livello di riempimento dei cassonetti e seguire percorsi calcolati per massimizzare l'efficienza della raccolta.
    \item \textbf{Dashboard di analisi}:
    \begin{itemize}
        \item Monitoraggio delle zone con pi\`u rapido riempimento dei cassonetti.
        \item Identificazione di aree da potenziare con nuovi cassonetti o ecopunti.
    \end{itemize}
    \item \textbf{Previsioni basate su dati storici}: analisi dei rifiuti in base a fattori stagionali, condizioni meteorologiche e dati degli anni passati.
    \item \textbf{Gestione dei cestini stradali}: monitoraggio e ottimizzazione della raccolta anche per i cestini pi\`u piccoli.
\end{itemize}

