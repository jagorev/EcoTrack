\section{Vincoli Object Constraint Language}

In questa sezione vengono specificati i vincoli espressi in linguaggio OCL (Object Constraint Language). Si includono invarianti, precondizioni e postcondizioni relative alle operazioni principali delle classi. I vincoli permettono di garantire la consistenza logica del sistema.

\subsection{Classe Autenticatore}

\textbf{Pre-Condizione \texttt{login()}:} L’username e la password non devono essere nulli o vuoti.
\begin{lstlisting}[language=OCL]
context Autenticatore::login(username: String, password: String): Boolean
pre: not username.oclIsUndefined() and username <> '' and
     not password.oclIsUndefined() and password <> ''
\end{lstlisting}

\textbf{Post-Condizione \texttt{login()}:} Se le credenziali sono corrette, l’utente risulta autenticato.
\begin{lstlisting}[language=OCL]
context Autenticatore::login(username: String, password: String): Boolean
post: result = true implies self.isAutenticato(self.idUtente) = true
\end{lstlisting}

\textbf{Invariante:} Un utente autenticato deve avere un ID valido (positivo).
\begin{lstlisting}[language=OCL]
context Autenticatore
inv: self.isAutenticato(self.idUtente) implies self.idUtente > 0
\end{lstlisting}

\subsection{Classe GestoreAutenticazioneEsterna}

\textbf{Post-Condizione \texttt{login()}:} L’utente viene autenticato solo se il provider conferma le credenziali.
\begin{lstlisting}[language=OCL]
context GestoreAutenticazioneEsterna::login(): Boolean
post: result = true implies self.isAutenticato(self.idUtente) = true
\end{lstlisting}


\subsection{Classe UtenteRegistrato}

\textbf{Pre-Condizione \texttt{setTelefono()}:} Il numero di telefono deve essere lungo almeno 9 cifre.
\begin{lstlisting}[language=OCL]
context UtenteRegistrato::setTelefono(tel: String): void
pre: tel.size() >= 9
\end{lstlisting}

\textbf{Post-Condizione \texttt{setUsername()}:} L’attributo username viene aggiornato.
\begin{lstlisting}[language=OCL]
context UtenteRegistrato::setUsername(un: String): void
post: self.getUsername() = un
\end{lstlisting}

\subsection{Classe GestoreSegnalazioni}

\textbf{Invariante:} Nessuna segnalazione può avere una data futura.
\begin{lstlisting}[language=OCL]
context GestoreSegnalazioni
inv: self.getSegnalazioni()->forAll(s | s.getData() <= Date::now())
\end{lstlisting}

\subsection{Classe Prenotazione}

\textbf{Invariante:} La data della prenotazione deve essere futura.
\begin{lstlisting}[language=OCL]
context Prenotazione
inv: self.getData() > Date::now()
\end{lstlisting}


\subsection{Classe UnitaRaccolta}

\textbf{Invariante:} Il livello di saturazione non può superare 100\%.
\begin{lstlisting}[language=OCL]
context UnitaRaccolta
inv: self.getLivelloSaturazione() <= 1.0
\end{lstlisting}

\textbf{Post-Condizione \texttt{calcolaSaturazione()}:} Il nuovo livello corrisponde al dato sensore/capienza.
\begin{lstlisting}[language=OCL]
context UnitaRaccolta::calcolaSaturazione(datoSensore: Real): void
post: self.getLivelloSaturazione() = datoSensore / self.getCapienza()
\end{lstlisting}

\textbf{Invariante:} Una unità è critica se supera l’85\% di saturazione.
\begin{lstlisting}[language=OCL]
context UnitaRaccolta
inv: self.isFull() = (self.getLivelloSaturazione() >= 0.85)
\end{lstlisting}

\subsection{Classe GestoreUtenti}

\textbf{Pre-Condizione \texttt{creaUtente()}:} L’ID non deve essere già utilizzato.
\begin{lstlisting}[language=OCL]
context GestoreUtenti::creaUtente(id: Integer): void
pre: self.utenti->forAll(u | u.getId() <> id)
\end{lstlisting}

